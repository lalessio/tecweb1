\documentclass[12pt]{article}
\usepackage{hyperref}
\usepackage{graphicx}
\usepackage[font=small,labelfont=bf]{caption}
\title{Progetto di fine corso}
\date{17/05/2016}
\author{Alessio Luca,Carlo Sindico}

\begin{document}
	\pagenumbering{arabic}
	
	\begin{titlepage}
		\newcommand{\HRule}{\rule{\linewidth}{0.5mm}}%linea orizzontale
		\center
		
		\textsc{\LARGE Universit\`a degli Studi di Padova}\\[1.5cm] 
		\textsc{\Large Laurea in Informatica}\\[0.5cm]
		\textsc{\large Corso di Tecnologie Web}\\[0.5cm]
		\textsc{\large Progetto di fine corso}\\[0.5cm]
		
		%----------------------------------------------------------------------------------------
		%	TITLE SECTION
		%----------------------------------------------------------------------------------------
		
		\HRule \\[0.4cm]
		{ \huge  Parco Naturale Monte Verde}\\[0.3cm] 
		\HRule \\[0.4cm]
		
		
		%----------------------------------------------------------------------------------------
		%	AUTHOR SECTION
		%----------------------------------------------------------------------------------------
		
		\begin{minipage}{0.3\textwidth}
			\begin{flushleft} \large
				\emph{Studente:}\\
				Luca \textsc{Alessio} % Your name
			\end{flushleft}
		\end{minipage}
		~
		\begin{minipage}{0.3\textwidth}
			\begin{flushright} \large
				\emph{Matricola:} \\
				\textsc{1070690} % Supervisor's Name
			\end{flushright}
		\end{minipage}\\[1cm]
		
			\begin{minipage}{0.3\textwidth}
				\begin{flushleft} \large
					\emph{Studente:}\\
					Carlo \textsc{Sindico} % Your name
				\end{flushleft}
			\end{minipage}
			~
			\begin{minipage}{0.3\textwidth}
				\begin{flushright} \large
					\emph{Matricola:} \\
					\textsc{1069322} % Supervisor's Name
				\end{flushright}
			\end{minipage}\\[1cm]
			
		%----------------------------------------------------------------------------------------
		%	INFORMATION WEBSITE
		%----------------------------------------------------------------------------------------
		
		\textsc{\Large Link al sito:}\\[0.2cm]	
		\textit{//tecnologie-web.studenti.math.unipd.it/tecweb/$\sim$csindico/}\\[1cm]
		
		\textsc{\Large Mail del referente:}\\[0.2cm]	
		\textit{sindycarlo@gmail.com}\\[1cm]
		
		%----------------------------------------------------------------------------------------
		%	DATI LOGIN
		%----------------------------------------------------------------------------------------
		
			\textsc{\Large Login Admin:}\\[0.2cm]
			\textsc{ Username:}\textit{ admin }\\[0.1mm]
			\textsc{ Password:}\textit{ admin }\\[0.1mm]
			
		\vfill
	\end{titlepage}
	
	\newpage
	\renewcommand{\contentsname}{Indice}
	\tableofcontents
	
	
	\newpage
	\pagenumbering{arabic}
	
	\section{Abstract}
	\begin{itemize}
		\item Il sito realizzato riguarda il fittizio Parco Naturale Monte Verde.\\ L'obiettivo principale che il portale monteverde.it si pone \`e quello di fornire tutte le informazioni possibili sul parco agli utenti visitatori. Il sito \`e in lingua italiana.
		
		\item L'utenza che accede al sito ha la possibilit\`a di visualizzare pagine informative che descrivono flora, fauna, storia e conformazione del parco e le attivit\`a che si possono svolgere al suo interno.

		\item Nel sito sono presenti anche pagine contenenti contenuto dinamico come gli orari e i prezzi d'ingresso al parco ed una pagina dedicata alle notizie.

		\item L'amministratore del sito ha la possibilit\`a di alterare prezzi ed orari del parco e di creare nuove notizie.\\ \\ Una dettagliata spiegazione sulle funzionalit\`a accessibili dall'amministratore \`e presente nella sezione "Amministrazione del sito".

		\item \textbf{Disclaimer:} Il Parco Naturale Monte Verde \`e un luogo puramente fantastico e senza alcuna corrispondenza nel mondo reale.\\ \\ La maggior parte dei testi contenuti nel nostro sito sono stati presi in prestito da http://www.pnab.it/ perch\`e, nonostante il Parco Naturale Monte Verde sia un luogo fittizio, ci \`e sembrato poco elegante riempire le pagine di "lorem ipsum" e allo stesso tempo la stesura di testo verosimile e originale avrebbe richiesto troppo tempo.\\ \\ Per quanto riguarda le immagini, esse sono state ricercate tramite Google Images, su richiesta del docente possiamo fornire la provenienza di ciascuna immagine (non includeremo tali dati in questa relazione in quanto superflui).\\ \\ Il logo \`e stato preso dal sito https://greenmountainbaptist.org/ .

	\end{itemize}

\newpage

\section{Materiale consegnato}
			
			 I file consegnati sono organizzati su quattro cartelle:
			\begin{itemize}
				\item cgi-bin: cartella nella quale sono presenti i file .cgi
				\item data: in questa cartella sono contenuti i file .xml ed i relativi .xsd
				\item public-html: contiene i file .html e le sotto-cartelle:
\begin{itemize}
\item css: cartella contenente i file .css
		\item		 images: cartella contenente tutte le foto del sito
			\item	 js: cartella contenente i vari script realizzati in JavaScript
\end{itemize}		
\item relazione: cartella contenente la presente relazione in formato .pdf e le immagini in essa contenute a dimensioni originali		 
			\end{itemize}	
			Ciascun file verr\`a esaminato in dettaglio nelle successive sezioni. 
			
			\newpage

\section{Struttura del sito}
Il sito \`e diviso nelle seguenti aree primarie: 
		\begin{itemize}
			\item \textbf{Home}: pagina di benvenuto, contiene una breve descrizione del parco e vari link alle sezioni principali del sito
			\item \textbf{Chi Siamo}: pagina puramente a scopo informativo, contiene un elenco degli avvenimenti importanti nella storia del parco, dall'inaugurazione al giorno d'oggi
			\item \textbf{Natura e Territorio}: altra pagina descrittiva, fornisce informazioni sulla conformazione territoriale del parco e su flora e fauna presenti in esso
			\item \textbf{News e Attivit\`a}: pagina divisa in due sezioni, un primo settore riguardante le attivit\`a che \`e possibile svolgere nel parco ed una seconda zona contente le notizie pi\`u recenti
			\item \textbf{Orari e Prezzi}: come suggerisce il nome, da questa pagina l'utente pu\`o conoscere orari e prezzi d'ingresso al parco
			\item \textbf{Info e Contatti}: contiene informazioni quali il regolamento del parco, le istruzioni per raggiungerlo e contatti utili
		\end{itemize}				
Da ogni pagina \`e possibile raggiungere la Mappa del Sito dal link posto prima del footer.	

\newpage

\section{Struttura delle pagine}
Le pagine del sito differiscono per contenuto ma condividono i seguenti elementi:
\begin{itemize}
\item \textbf{Header}: la parte superiore di ogni pagina contiene il logo (che linka alla home) e il nome del sito in alto a sinistra, subito sotto sono invece presenti dei collegamenti alle aree principali del sito (vedi Struttura del sito). L'header pu\`o essere saltato da un utente che utilizza uno screen reader tramite apposite ancore d'aiuto (vedi sezione Accessibilit\`a)
\item \textbf{Breadcrumb}: rende noto all'utente la sua posizione precisa all'interno del sito, dal breadcrumb \`e possibile risalire alle pagine precedentemente visitate
\item \textbf{Link d'aiuto}: in ogni pagina \`e sempre presente un link alla Mappa del Sito e l'ancora "Torna all'Inizio" che riporta la visuale all'inizio della pagina, utile nelle pagine con contenuto corposo
\item \textbf{Footer}: contiene brevi informazioni come l'indirizzo del parco e le certificazioni di aderenza agli standard del sito; dal footer \`e possibile loggare/sloggare come amministratore
\end{itemize}

\newpage

\section{Amministrazione del sito}
\textit{Login:} admin \textit{Password:} admin
\begin{itemize}
\item \`E possibile accedere all'area amministrativa da qualsiasi pagina tramite un link posto nel footer.
\item L'amministratore ha la possibilit\`a di modificare gli orari e i prezzi d'ingresso al parco utilizzando i form appositi. 
\item I form presentano delle liste a cascata da cui \`e possibile scegliere il preciso dato da modificare (una lista con i giorni della settimana per gli orari, due liste con tipologia e periodo di validit\`a per il prezzo del biglietto) e un campo testuale in cui inserire i nuovi valori.\\ \\ Il form per il cambio dell'orario accetta qualsiasi input testuale (perch\`e sono permessi sia valori misti come "8:30 - 18:30" sia valori puramente testuali come "Chiuso") mentre il prezzo del biglietto dovr\`a necessariamente essere un intero.
\item L'amministratore pu\`o pubblicare nuove notizie riguardanti eventi, promozioni e attivit\`a inerenti al parco. Di una notizia \`e necessario conoscere il titolo, il corpo dell'articolo e un' immagine ad essa associata (oltre alla data che per\`o viene automaticamente memorizzata dal sistema senza dover essere inserita manualmente).
\item L'amministratore pu\`o eliminare individualmente le varie notizie recandosi nella pagina "Archivio News" (News e Attivit\`a $\rightarrow$ Archivio News link in basso) e cliccando su "Elimina". Alternativamente \`e possibile eliminare una notizia direttamente nella pagina che la visualizza.
\item Quando l'amministratore \`e loggato il footer cambia (vedi footer\_manager.js) permettendo il logout tramite l'apposito pulsante o il ritorno all'area amministratore in caso ci si trovi in un'altra pagina.
\end{itemize}

\newpage

			\section{HTML DA FARE}

 La struttura HTML del sito viene interamente "stampata" da vari file .cgi: questa scelta \`e dovuta alla grande presenza di contenuto dinamico all'interno del sito per cui non sono presenti file .html "puri" se non index.html che \`e per\`o un semplice redirect a menu.cgi, file Perl che effettivamente stampa la homepage. \\ \\
Le pagine HTML del sito (stampate dal cgi) sono state realizzate rispettando lo standard XHTML 1.0 Strict.


			\section{Perl DA FARE}
		
				  Le pagine scritte in Perl si dividono principalmente in due tipologie: pagine "dinamiche" di rappresentazione e pagine di elaborazione dei dati.\\ \\
				Alla prima tipologia appartengono i file .cgi che eseguono il "print" della pagina html con il contenuto richiesto (ne \`e un esempio la pagina page\_template.cgi che si occupa della stampa a video di ogni ricetta) mentre le pagine della seconda tipologia sono solitamente "pagine di servizio" ovvero codice che esegue operazioni dietro le quinte come salvataggi ed eliminazioni di dati sui file xml. \\ \\Nel dettaglio, questo \`e ci\`o di cui ciascun file si occupa:
\begin{itemize}

				\item  menu.cgi - pagina principale del sito, contiene del contenuto dinamico in quanto le "ricette consigliate" che compaiono nella pagina sono casuali e variano ad ogni accesso alla pagina (inoltre in ogni pagina del sito il footer varia leggermente a seconda se l'amministratore sia loggato oppure meno).
				
				\item  antipasti.cgi, secondi.cgi, primo.cgi, dessert.cgi - semplici sottomen\`u interni che si occupano di raggruppare le ricette appartenenti alla stessa categoria per facilitare la navigazione dell'utente

				\item page\_template.cgi - pagina dinamica che visualizza la ricetta richiesta, funziona nel seguente modo: il link cliccato dall'utente che lo indirizza a questa pagina contiene un parametro (id) differente per ogni ricetta, page\_template.cgi cerca il parametro passato tra gli attributi IDCode delle singole ricette nel file 4forchette.xml (vedi sezione XML) ed identifica quale era la ricetta richiesta di cui va poi a restituirne le informazioni su schermo.

				\item proponiricetta.cgi - stampa la pagina "Proponi una ricetta" attraverso la quale l'utente ha la possibilit\`a di proporre nuove ricette da inserire nel sito. La pagina contiene un form di discrete dimensioni e delle istruzioni sul suo completamento. Un'analisi pi\`u dettagliata su come avviene il processo di creazione, memorizzazione e approvazione delle ricette verr\`a effettuata in un successivo paragrafo.

				\item handle\_proposta.cgi - quando l'utente ha compilato correttamente (vedi sezione Javascript) il form per la proposta di una nuova ricetta, la ricetta viene salvata nel "database" xml. La nuova ricetta non verr\`a tuttavia ancora visualizzata nel sito, prima dovr\`a infatti venire approvata dall'amministratore (vedi accept\_ricetta.cgi).

				\item contatti.cgi - stampa la pagina dei contatti, qui l'utente ha la possibilità di vedere i messaggi lasciati da altri utenti ed eventualmente pu\`o lasciarne uno proprio compilando il form apposito dove viene richiesto un nome ed il testo effettivo del commento (sarebbe stato interessante implementare un sistema di registrazione e gestione degli utenti nel sito ma date le scarse dimensioni del gruppo e il tempo disponibile si \`e optato per una soluzione meno impegnativa).

				\item inserisci\_commento.cgi - gestisce l'inserimento di un nuovo commento traducendo l'input dell'utente in dati xml.

				\item amministratore\_login.cgi - contiene il form che l'amministratore utilizza per effettuare l'accesso all'area riservata. Questa pagina \`e accessibile solamente dal link posto nel footer della pagina principale (menu.cgi).

				\item controllo\_login.cgi - controlla la correttezza delle credenziali inserite nel form del punto precedente. Al momento \`e previsto un solo amministratore per il sistema. L'amministratore viene identificato come autenticato quando un parametro apposito (\$auth) \`e impostato ad un preciso valore e ci\`o viene settato in degli appositi cookie (vedi codice del file per i dettagli) per rendere l'informazione nota tra le diverse pagine.

				\item console\_admin.cgi - stampa l'hub amministrativo da cui \'e possibile gestire semplicemente l'intero sito. Da questa pagina l'amministratore ha la possibilit\`a di accettare o rifiutare le nuove ricette, eliminare vecchie ricette ed eliminare eventuali commenti ritenuti offensivi.

				\item accept\_ricetta.cgi - concede ad una ricetta il permesso di venire visualizzata nel sito (semplicemente il suo attributo accepted viene settato positivamente).

				\item delete\_ricetta.cgi - elimina una ricetta dal file xml. Operazione irreversibile.

				\item delete\_commento.cgi - elimina un commento dal file xml. Operazione irreversibile.

				\item logout.cgi - breve script che effettua il logout dell'amministratore. Viene invocato alla pressione del tasto "logout" visibile sul footer di ogni pagina qualora l'amministratore sia attualmente autenticato nel sistema.

				\item cercaricetta.cgi - nell'header di ogni pagina è presente una barra di ricerca, qualora l'utente digitasse una stringa di caratteri e ne richiedesse l'elaborazione, questo \`e lo script che verrebbe eseguito: cercaricetta.cgi stampa una pagina contente un elenco di tutte le ricette il cui titolo matcha in parte o totalmente il parametro di ricerca inserito dall'utente. La ricerca avviene semplicemente scorrendo l'xml isolandone nomi ed id di tutte le ricette e restituendo solamente quelle che contengono una corrispondenza.

				\item funzioni.pl  contiene funzioni di servizio usate in diversi contesti (ad es. trim degli input).	

\end{itemize}	
			
		\newpage	
	\section{CSS DA FARE}
			\begin{itemize}
				\item Nella realizzazione dell'interfaccia grafica del sito è stato usato lo standard CSS3.
				\item Allo stesso tempo si \'e fatta molta attenzione alla compatibilit\'a con browser pi\'u datati, e si \'e cercato di utilizzare un numero ristretto delle nuove funzionalit\'a offerte dallo standard.
				
				\item Alcune delle funzionalit\'a CSS3 che sono state utilizzate:
				Border-radius e Box shadow: per realizzare i pulsanti delle form , e per le immagini.
				
			
				\item Nella cartella public-html/css sono presenti i seguenti fogli di stile:

				\subitem style.css: modella lo stile di visualizzazione del sito sia per gli utenti desktop (che hanno uno schermo largo al massimo 1200px) che per gli utenti mobile con dispositivi con schermo piccolo (che hanno uno schermo largo massimo 390px) e dispositivi con dimensioni schermo intermedie ( max-width=730px).


				\subitem print.css: modella lo stile di stampa delle pagine del sito.(particolare attenzione si \'e data alla stampa delle pagine delle ricette).

\end{itemize}
					\newpage
				
		\section{XML DA FARE}
		Sono presenti 3 file xml principali con i rispettivi xsd:

		\begin{itemize}
		\item  amministratore.xml - semplice file ausiliario per il controllo del corretto login nell'area amministrativa, contiene solamente due campi per il login e la password.
		
		\item commenti.xml - contiene i commenti contenuti nella sezione contatti; dalla console amministrativa è possibile intervenire indirettamente sul file attuando operazioni di eliminazione
		
		\item 4forchette.xml - cuore centrale del sistema, questo documento xml svolge la funzione di database per le ricette: anche qui l'amministratore può effettuare operazioni di eliminazione mentre l'inserimento è riservato all'utente (questo aspetto \`e esaminato nel dettaglio nelle sezioni successive)
		\end{itemize}				
					
			\section{Javascript DA FARE}
			\begin{itemize}
				

				\item Javascript \'e stato utilizzato principalmente per il controllo degli input dei form. 
				\item Nel sito sono presenti 3 differenti form, quello per il login nell'area amministrativa, un secondo per il submit dei commenti nella pagina dei contatti ed infine il pi\`u complesso nella sezione per la proposta di nuove ricette. 
				\item Per quanto riguarda i primi due form (gestiti rispettivamente dai file controllo\_login.js e valida\_commento.js), una volta che l'utente clicca su "Submit" vengono analizzati in ordine di apparizione tutti i campi del form, se ne viene identificato uno vuoto allora l'invio viene bloccato e l'utente ne viene notificato da un messaggio d'errore specifico per quel campo.
				\item Per quanto riguarda il form "Proponi ricetta" invece (gestito dal file proponi\_ricetta.js), oltre a questi controlli basilari ne viene effettuato un altro pi\`u approfondito nell'area di inserimento degli ingredienti. 
				\item Dato che ogni ricetta presenta un numero di ingredienti sempre differente era impossibile prevedere un numero definito di campi per ciascun ingrediente nel form (in quanto sarebbero stati spesso insufficienti oppure troppo numerosi e comunque sgradevoli visivamente), per risolvere questo problema si \`e inizialmente pensato di realizzare un form dinamico che aggiungesse campi su richiesta dell'utente. Questa soluzione \`e stata però scartata per difficolt\`a tecniche nella realizzazione e principalmente per il fatto che non fosse accessibile agli screen reader. 
				\item Si \`e optato quindi per lasciare la zona di inserimento degli ingredienti come una semplice area di testo all'interno della quale per\`o vengono applicate delle regole per l'identificazione dei singoli ingredienti: dopo ogni ingrediente \`e necessario inserire il carattere di separazione ";" (punto e virgola senza virgolette) ed andare a capo. Se il testo inserito dall'utente rispetta questa sintassi l'input verr\`a accettato, altrimenti verr\`a fornito un messaggio d'errore. Nella pagina sono presenti delle istruzioni sull'utilizzo del form che spiegano anche l'appropriato inserimento degli ingredienti; riteniamo che la maggior parte degli utenti non trovi difficolt\`a ad usare questa sintassi che \`e inoltre pienamente accessibile ed utilizzabile da utenti non vedenti.		
		
			\end{itemize}
	\newpage
	Come interagiscono le varie tecnologie tra di loro? Vediamo un esempio...
	\section{Spiegazioen archivio news? forse la includo in un altra sezione non so ancora}
			

\newpage

	\newpage
		\section{Accessibilit\`a DA FARE MEGLIO STAVOLTA}
		\subsection{Separazione tra struttura,presentazione e comportamento}
		\begin{itemize}
			\item Per una maggiore accessibilit\'a al sito da parte di utenti disabili e per favorire gli algoritmi dei motori di ricerca si \'e deciso di separarare la struttura dalla presentazione e dal comportamento.
			Infatti il contenuto del sito \'e rapprentato dai file HTML e CGI, i quali richiamano i fogli di stile CSS e si utilizzano (anche in questo caso attraverso percorsi esterni),controlli in JavaScript in particolare per la compilazione dei form. 

			\item Il contenuto rimane accessibile anche se JavaScript \'e disabilitato. Infatti opportuni controlli in Perl ne verificano la validit\`a.

			\item Tutto il codice \'e stato scritto seguendo le disposizioni W3C con opportuna validazione sui loro validatori.
		\end{itemize}
			\subsection{Colori}
			\begin{itemize}
				\item Si \'e scelto uno schema di colori non particolarmente vivace (un mix di azzuri chiaro); anche se non sono colori di base comunque la lettura dei testi risulta accessibile. Per un sito di cucina inoltre è opportuno non utilizzare colori troppi aggressivi e decisi ma attenersi ad uno stile più sobrio e rilassato. 

				\item I link sono sempre sottolineati, e diventano di colore viola quando vengono cliccati.

			Di seguito sono riportate le visualizzazioni del sito attraverso alcuni disturbi visivi:

			\begin{figure}
			\centering
			\includegraphics[width=90mm]{deuteranopia}
			\caption{vedi file deuteranopia.png}
			\end{figure} 

			\begin{figure}
			\centering
			\includegraphics[width=90mm]{protanopia}
			\caption{vedi file protanopia.png}
			\end{figure}

			\begin{figure}
			\centering
			\includegraphics[width=90mm]{tritanopia}
			\caption{vedi file tritanopia.png}
			\end{figure}
			
			\end{itemize}	

			
			\newpage
			\subsection{Tag meta}
			
	 Sono stati inseriti per ogni pagina i tag meta:
	 		\begin{itemize}
				\item title: descrive la pagina corrente dal particolare al generale.
				\item description: da una descrizione del contenuto del sito
				\item keywords: contiene parole chiave per i motori di ricerca
				\item language: indica che il sito \'e stato interamente scritto in italiano.
				\item author: indica l'autore/i del sito
				\item content-type: contiene direttive per il browser
				\item viewport: esprime indicazioni per la visualizzazione
			\end{itemize}
			\subsection{Screen reader}
			\begin{itemize}
				\item Ogni foto ha il suo attributo alt che descrive ci\'o che l'immagine ritrae.
				Si \`e evitato di utilizzare immagini per visualizzare il testo, quindi il contenuto informativo rimane accessibile anche quando fallisce il caricamento delle immagini o del CSS.
				\item Si \'e fornita particolare attenzione alle parole straniere che sono state segnalate agli screenreader attraverso il tag "span xml:lang=en" segnalando la lingua con cui leggere correttamente i vocaboli. 
				\item Inoltre, nelle pagine contenenti form, \`e stato inserito un link skip nav per saltare direttamente al contenuto qualora l'utente dello screen reader non voglia riascoltare nuovamente il men\`u
			\end{itemize}
\newpage			\section{Usabilit\`a DA FARE MEGLIO STAVOLTA}
			L'utente riesce ad orientarsi nel sito? Analizziamo alcuni degli assi principali del giornalismo. Ometteremo gli assi Who e When dato che nel nostro contesto non \`e necessario dare all'utente informazioni su chi c'\`e dietro al sito (\`e un progetto didattico) n\`e sono presenti novità continue.
			\begin{itemize}
				
				\item What?: 
				Un utente appena entra nella home capisce subito che si tratta di un sito di ricette, dalla barra dei men\'u (Proponi ricetta, Cerca ricetta), e dal contenuto in primo piano che mette in evidenza alcune ricette proposte.
				
				\item Where?: 
				L'utente riesce sempre a capire dove si trova grazie al breadcrumb, inoltre l'header costituisce un importante punto di riferimento per la navigazione dell'utente.
				
				\item Why?: 
				Perch\'e un utente dovrebbe rimanere nel sito o dovrebbe ritornarci? Il sito \'e principalmente espositivo,(gli utenti possono liberamente visualizzare le ricette), si \'e cercato di renderlo pi\'u interessante, aggiungendo una sezione proponi ricetta (l'utente ha la possibilit\'a di inserire la propria ricetta), e anche una sezione commenti.
				
				\item How?:
				 La barra di navigazione mostra tutte le sezioni principali del sito alle quali un utente pu\'o accedere.
				Nella barra men\'u \'e sempre evidenziata la voce della pagina in cui ci troviamo,e si vede attraverso una diversa colorazione dei link in quali altre pagine si \'e stati. 

				
			\end{itemize}
	
	
\end{document}